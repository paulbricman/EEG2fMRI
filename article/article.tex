\documentclass{article}
\usepackage[utf8]{inputenc}
\usepackage{authblk}

\title{EEG2fMRI: Cross-Modal Image Synthesis for Functional Neuroimaging}
\author{Paul Bricman}
\author{Jelmer Borst}
\affil{University of Groningen}
\date{July 2020}

\begin{document}

\maketitle

\section{Introduction}

Neural activity has been extensively used for studying cognition.

Due to the increased relevance of neural activity in cognitive science, numerous functional neuroimaging techniques have been developed.

Given the lack of techniques which individually exhibit high performance across all the aforementioned metrics, there have been attempts at using multiple techniques at once.

Besides multi-modal functional neuroimaging, another proposed approach for bypassing the limitations of individual techniques has been uni-modal image synthesis. This task consists of synthetizing data collected through a modality based on data already collected from that modality.

Despite the success of multi-modal functional neuroimaging in several applications, limited attention has been given to the task of cross-modal image synthesis. This task consists of reconstructing data collected through a target modality based on data collected through a source modality, by exploiting the inherent relation between them. In a setting where both source and target modalities are available, this task might have limited utility. However, the value of cross-modal image synthesis comes from settings were only one modality is available. In such settings, data collected through an unavailable modality can be approximated.



\end{document}
