\documentclass{article}
\usepackage[utf8]{inputenc}
\usepackage{authblk}

\title{EEG2fMRI: Cross-Modal Image Synthesis for Functional Neuroimaging}
\author{Paul Bricman}
\author{Jelmer Borst}
\affil{University of Groningen}
\date{July 2020}

\begin{document}

\maketitle

\section{Introduction}

Neural activity has been extensively used for investigating cognition.

Due to the increased relevance of neural activity in cognitive science, a large number of functional neuroimaging techniques have been developed.

Given the lack of techniques which individually exhibit high overall performance, investigators have attempted to integrate their complementary qualities by using multiple techniques simultaneously.

Besides multi-modal functional neuroimaging, investigators have also proposed uni-modal image synthesis as a method of augmenting neuroimaging techniques. This task consists of synthetizing novel data collected through a modality based on data previously collected from that modality.

In addition to uni-modal image synthesis, investigators have also proposed cross-modal image synthesis as a method of augmenting neuroimaging techniques. This task consists of reconstructing data collected through a target modality based on data collected through a distinct source modality, by exploiting the inherent relation between them. In a setting where both source and target modalities are available, this task might have limited utility. However, the value of cross-modal image synthesis comes from settings where only one modality is available. In such settings, data collected through an unavailable modality can be approximated. Unfortunately, to the best of the authors' knowledge, this approach has only been used for augmenting structural neuroimaging, with no analogous efforts for functional neuroimaging.

Despite the major demand for high-performance functional neuroimaging and the success of cross-modal image synthesis for structural neuroimaging, limited attention has been given to cross-modal image synthesis for functional neuroimaging.



\end{document}
